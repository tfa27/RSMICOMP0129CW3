% \newpage
\section{Methodology}\label{Sec:intro}
%%%%%%%%%%%%%%%%%%%%%%%%%%%%%%%%%%%%%%%%%%%%%%%%%%%%%%%%%%%%%%%%%%%%%%%%
\subsection*{Visual}\label{Sec:methvis}
In order to extract the pose of the target cylinder from the RGB-D point cloud generated by the camera, it is necessary to separate the points corresponding to the cylinder from the rest of the points. Knowing that the cylinder is the only cylindrical object in the point cloud, RANSAC can be used to identify the points corresponding to an infinite cylindrical model based on their position and surface normals. From the fitted model the cylinder pose and radius can be extracted. The cylinder height and centre point can then be found using the lowest and highest points of the cylinder and the camera pose.

However, the point cloud's large number of data-points can slow down computation of the cylinder pose. To fix this, two filters where implemented. The first one was a downsampling filter, which applied a voxel grid of a specified size over the entire point cloud and merged the multiple points in each voxel into a single, averaged one. The second was a passthrough filter, which involved removing all points outside of the approximate area the cylinder was expected to lie in. These two approaches removed the number of points in the pointcloud and sped up the RANSAC computation, at the cost of a lower resolution (from the downsampling filter) leading to a lower accuracy in the cylinder pose.
%%%%%%%%%%%%%%%%%%%%%%%%%%%%%%%%%%%%%%%%%%%%%%%%%%%%%%%%%%%%%%%%%%%%%%%%
\subsection*{Motion}
In order to perform motion planning for this task the MoveIt ROS library is used. The cylinder pose is extracted using the methods described in the section above, and then MoveIt is used to perform inverse kinematics to find the robot configurations necessary to approach, grasp, retreat, and place the cylinder and the cuboid object.

The version of MoveIt used for this task uses iterative inverse kinematics (IIK) rather than any closed form solutions. This is less than ideal due to the fact IIK is much less consistent than closed form solutions and can sometimes not come up with any solution at all. IIK can also be very slow depending on the current manipulator configuration and the desired one. The cuboid pose and the pose of the tables are acquired from the MoveIt simulation. 
